
%% The first command in your LaTeX source must be the \documentclass command.
%%
%% Options:
%% twocolumn : Two column layout.
%% hf: enable header and footer.
\documentclass[
% twocolumn,
% hf,
]{ceurart}
\usepackage{amsmath,amssymb,longtable,hhline}
\usepackage{doi}
\def\doitext{DOI:}

%%
%% One can fix some overfulls
\sloppy

%%
%% Minted listings support
%% Need pygment <http://pygments.org/> <http://pypi.python.org/pypi/Pygments>
% \usepackage{listings}
\usepackage{minted}
\setminted{fontsize=\footnotesize,mathescape}
\usepackage{hyperref}

\hypersetup{
    bookmarks=true,         % show bookmarks bar?
    unicode=true,           % non-Latin characters in Acrobat’s bookmarks
    pdftoolbar=false,        % show Acrobat’s toolbar?
    pdfmenubar=false,        % show Acrobat’s menu?
    pdffitwindow=false,     % window fit to page when opened
    pdfstartview={FitH},    % fits the width of the page to the window
    pdftitle={},    % title
    pdfauthor={Evgeny Cherkashin},     % author
    pdfsubject={model driven architecture},   % subject of the document
    pdfnewwindow=true,      % links in new PDF window
    colorlinks=true,       % false: boxed links; true: colored links
    linkcolor=red,          % color of internal links (change box color with linkbordercolor)
    citecolor=green,        % color of links to bibliography
    filecolor=magenta,      % color of file links
    urlcolor=blue           % color of external links
  }

\usepackage{pifont}
\graphicspath{{pics/}}


%% auto break lines
% \lstset{breaklines=true}

%%
%% end of the preamble, start of the body of the document source.
\begin{document}

%%
%% Rights management information.
%% CC-BY is default license.
\copyrightyear{2021}
\copyrightclause{Copyright for this paper by its authors.
  Use permitted under Creative Commons License Attribution 4.0
  International (CC BY 4.0).}

%%
%% This command is for the conference information
\conference{2nd International Workshop on Advanced Information and Computation Technologies and Systems, December XX--XX, 2021, Irkutsk, Russia}

%%
%% The "title" command
\title{Technologies of Semantic WEB as an environment of application development and integration}
%%
%% The "author" command and its associated commands are used to define
%% the authors and their affiliations.
\author[1,2]{Evgeny A. Cherkashin}[%
orcid=0000-0003-2428-2471,
email=eugeneai@icc.ru,
url=https://github.org/eugeneai,
]
\address[1]{Matrosov Institute for System Dynamics and Control Theory of Siberian Branch of Russian Academy of Sciences, 134 Lermontov St, Irkutsk, 664033, Russian Federation}

\address[2]{Institute for Mathematics and Information Technologies, Irkutsk State University, 20~Gagarina Bulv, Irkutsk, 664003, Russian Federation}

%%
%% The abstract is a short summary of the work to be presented in the
%% article.
\begin{abstract}
  Semantic Web technologies and standards are becoming one of productive assets for software development and integration on various levels of design and implementation.  This include OSI level 6 data representation, database and file storage formats, the source data for indexes, hypertext markup, source data for document authoring, representing abstract software models, \emph{etc}.  They also smoothly combined with development and data analysis tools, giving raise a new common-ground environment for realizing integration on the semantic level.

  This paper presents the view on the Semantic Web technologies as data representation for software development tools with model analysis, transformation and retrospection capabilities.  Three examples are presented, exposing our experience.
\end{abstract}

%%
%% Keywords. The author(s) should pick words that accurately describe
%% the work being presented. Separate the keywords with commas.
\begin{keywords}
  knowledge graph \sep
  semantic web \sep
  model transformation \sep
  model-driven architecture \sep
  object-oriented logical programming \sep
  web application \sep
  geographical information system
\end{keywords}


%%
%% This command processes the author and affiliation and title
%% information and builds the first part of the formatted document.
\maketitle

\section{Introduction}

CEUR-WS's article template provides a consistent \LaTeX{} style for
use across CEUR-WS publications, and incorporates accessibility and
metadata-extraction functionality. This document will explain the
major features of the document class\footnote{You can use this
  document as the template for preparing your publication. We
  recommend using the latest version of the ceurart style.}.

If you are new to publishing with CEUR-WS, this document is a valuable
guide to the process of preparing your work for publication.

The ``\verb|ceurart|'' document class can be used to prepare articles
for any CEUR-WS publication, and for any stage of publication, from
review to final ``camera-ready'' copy with {\itshape very} few changes
to the source.

This class depends on the following packages
for its proper functioning:

\begin{itemize}
\item \verb|natbib.sty| for citation processing;
\item \verb|geometry.sty| for margin settings;
\item \verb|graphicx.sty| for graphics inclusion;
\item \verb|hyperref.sty| optional package if hyperlinking is required in
  the document;
\item \verb|fontawesome5.sty| optional package for bells and whistles.
\end{itemize}

All the above packages are part of any
standard \LaTeX{} installation.
Therefore, the users need not be
bothered about downloading any extra packages.

\section{Modifications}

Modifying the template --- including but not limited to: adjusting
margins, typeface sizes, line spacing, paragraph and list definitions,
and the use of the \verb|\vspace| command to manually adjust the
vertical spacing between elements of your work --- is not allowed.

\section{Template parameters}

There are a number of template
parameters which modify some part of the \verb|ceurart| document class.
This parameters are enclosed in square
brackets and are a part of the \verb|\documentclass| command:
% \begin{lstlisting}
%   \documentclass[parameter]{ceurart}
% \end{lstlisting}

Frequently-used parameters, or combinations of parameters, include:
\begin{itemize}
\item {\verb|twocolumn|}: Two column layout.
\item {\verb|hf|}: Enable header and footer\footnote{You can enable
    the display of page numbers in the final version of the entire
    collection. In this case, you should adhere to the end-to-end
    pagination of individual papers.}.
\end{itemize}

\section{Front matter}

\subsection{Title Information}

The titles of papers should be either all use the emphasizing
capitalized style or they should all use the regular English (or
native language) style. It does not make a good impression if you or
your authors mix the styles.

Use the \verb|\title| command to define the title of your work. Do not
insert line breaks in your title.

\subsection{Title variants}

% \verb|\title| command have the below options:
% \begin{itemize}
% \item \verb|title|: Document title. This is default option.
% \begin{lstlisting}
% \title[mode=title]{This is a title}
% \end{lstlisting}
% You can just omit it, like as follows:
% \begin{lstlisting}
% \title{This is a title}
% \end{lstlisting}

% \item \verb|alt|: Alternate title.
% \begin{lstlisting}
% \title[mode=alt]{This is a alternate title}
% \end{lstlisting}

% \item \verb|sub|: Sub title.
% \begin{lstlisting}
% \title[mode=sub]{This is a sub title}
% \end{lstlisting}
% You can just use \verb|\subtitle| command, as follows:
% \begin{lstlisting}
% \subtitle{This is a sub title}
% \end{lstlisting}

% \item \verb|trans|: Translated title.
% \begin{lstlisting}
% \title[mode=trans]{This is a translated title}
% \end{lstlisting}

% \item \verb|transsub|: Translated sub title.
% \begin{lstlisting}
% \title[mode=transsub]{This is a translated sub title}
% \end{lstlisting}
% \end{itemize}

\subsection{Authors and Affiliations}

Each author must be defined separately for accurate metadata
identification. Multiple authors may share one affiliation. Authors'
names should not be abbreviated; use full first names wherever
possible. Include authors' e-mail addresses whenever possible.

\verb|\author| command have the below options:

\begin{itemize}
\item \verb|style|: Style of author name (chinese)
\item \verb|prefix|: Prefix
\item \verb|suffix|: Suffix
\item \verb|degree|: Degree
\item \verb|role|: Role
\item \verb|orcid|: ORCID
\item \verb|email|: E-mail
\item \verb|url|: URL
\end{itemize}

Author names can have some kinds of marks and notes:
\begin{itemize}
\item affiliation mark: \verb|\author[<num>]|.
% \item email: \verb|\ead{<email>}|,
% \item url: \verb|\ead[url]{<url>}|.
\end{itemize}

The author names and affiliations could be formatted in two ways:
\begin{enumerate}
\item Group the authors per affiliation.
\item Use an explicit mark to indicate the affiliations.
\end{enumerate}

Author block example:
% \begin{lstlisting}
% \author[1,2]{Author Name}[%
%     prefix=Prof.,
%     degree=D.Sc.,
%     role=Researcher,
%     orcid=0000-0000-000-0000,
%     email=name@example.com,
%     url=https://name.example.com
% ]

% \address[1]{Affiliation #1}
% \address[2]{Affiliation #2}
% \end{lstlisting}

% \subsection{Abstract and Keywords}

% Abstract shall be entered in an environment that starts
% with \verb|\begin{abstract}| and ends with
% \verb|\end{abstract}|.

% \begin{lstlisting}
% \begin{abstract}
%   This is an abstract.
% \end{abstract}
% \end{lstlisting}

% The key words are enclosed in a \verb|keywords|
% environment. Use \verb|\sep| to separate keywords.

% \begin{lstlisting}
% \begin{keywords}
%   First keyword \sep
%   Second keyword \sep
%   Third keyword \sep
%   Fourth keyword
% \end{keywords}
% \end{lstlisting}

At the end of front matter add \verb|\maketitle| command.

\section{Sectioning Commands}

Your work should use standard \LaTeX{} sectioning commands:
\verb|\section|, \verb|\subsection|,
\verb|\subsubsection|, and
\verb|\paragraph|. They should be numbered; do not remove
the numbering from the commands.

Simulating a sectioning command by setting the first word or words of
a paragraph in boldface or italicized text is not allowed.

\section{Tables}

The ``\verb|ceurart|'' document class includes the ``\verb|booktabs|''
package --- \url{https://ctan.org/pkg/booktabs} --- for preparing
high-quality tables.

Table captions are placed \textit{above} the table.

Because tables cannot be split across pages, the best placement for
them is typically the top of the page nearest their initial cite.  To
ensure this proper ``floating'' placement of tables, use the
environment \verb|table| to enclose the table's contents and the
table caption. The contents of the table itself must go in the
\verb|tabular| environment, to be aligned properly in rows and
columns, with the desired horizontal and vertical rules.

Immediately following this sentence is the point at which
Table~\ref{tab:freq} is included in the input file; compare the
placement of the table here with the table in the printed output of
this document.

\begin{table*}
  \caption{Frequency of Special Characters}
  \label{tab:freq}
  \begin{tabular}{ccl}
    \toprule
    Non-English or Math&Frequency&Comments\\
    \midrule
    \O & 1 in 1,000& For Swedish names\\
    $\pi$ & 1 in 5& Common in math\\
    \$ & 4 in 5 & Used in business\\
    $\Psi^2_1$ & 1 in 40,000& Unexplained usage\\
  \bottomrule
\end{tabular}
\end{table*}

To set a wider table, which takes up the whole width of the page's
live area, use the environment \verb|table*| to enclose the table's
contents and the table caption.  As with a single-column table, this
wide table will ``float'' to a location deemed more
desirable. Immediately following this sentence is the point at which
Table~\ref{tab:commands} is included in the input file; again, it is
instructive to compare the placement of the table here with the table
in the printed output of this document.

\begin{table}
  \caption{Some Typical Commands}
  \label{tab:commands}
  \begin{tabular}{ccl}
    \toprule
    Command &A Number & Comments\\
    \midrule
    \texttt{{\char'134}author} & 100& Author \\
    \texttt{{\char'134}table}& 300 & For tables\\
    \texttt{{\char'134}table*}& 400& For wider tables\\
    \bottomrule
  \end{tabular}
\end{table}

\section{Math Equations}

You may want to display math equations in three distinct styles:
inline, numbered or non-numbered display.  Each of the three are
discussed in the next sections.

\subsection{Inline (In-text) Equations}

A formula that appears in the running text is called an inline or
in-text formula.  It is produced by the \verb|math| environment,
which can be invoked with the usual
\verb|\begin| \ldots \verb|\end| construction or with
the short form \verb|$| \ldots \verb|$|. You can use any of the symbols
and structures, from $\alpha$ to $\omega$, available in
\LaTeX~\cite{Lamport:LaTeX};
this section will simply show a few
examples of in-text equations in context. Notice how this equation:
\begin{math}
  \lim_{n\rightarrow \infty} \frac{1}{n} = 0,
\end{math}
set here in in-line math style, looks slightly different when
set in display style.  (See next section).

\subsection{Display Equations}

A numbered display equation---one set off by vertical space from the
text and centered horizontally---is produced by the \verb|equation|
environment. An unnumbered display equation is produced by the
\verb|displaymath| environment.

Again, in either environment, you can use any of the symbols and
structures available in \LaTeX{}; this section will just give a couple
of examples of display equations in context.  First, consider the
equation, shown as an inline equation above:
\begin{equation}
  \lim_{n\rightarrow \infty} \frac{1}{n} = 0.
\end{equation}
Notice how it is formatted somewhat differently in
the \verb|displaymath|
environment.  Now, we'll enter an unnumbered equation:
\begin{displaymath}
  S_{n} = \sum_{i=1}^{n} x_{i} ,
\end{displaymath}
and follow it with another numbered equation:
\begin{equation}
  \lim_{x \to 0} (1 + x)^{1/x} = e
\end{equation}
just to demonstrate \LaTeX's able handling of numbering.

\section{Figures}

The ``\verb|figure|'' environment should be used for figures. One or
more images can be placed within a figure. If your figure contains
third-party material, you must clearly identify it as such, as shown
in the example below.
\begin{figure}
  \centering
  % \includegraphics[width=\linewidth]{sample-franklin}
  \caption{1907 Franklin Model D roadster. Photograph by Harris \&
    Ewing, Inc. [Public domain], via Wikimedia
    Commons. (\url{https://goo.gl/VLCRBB}).}
\end{figure}

Your figures should contain a caption which describes the figure to
the reader. Figure captions go below the figure. Your figures should
also include a description suitable for screen readers, to
assist the visually-challenged to better understand your work.

Figure captions are placed below the figure.

\section{Citations and Bibliographies}

The use of Bib\TeX{} for the preparation and formatting of one's
references is strongly recommended. Authors' names should be complete
--- use full first names (``Donald E. Knuth'') not initials
(``D. E. Knuth'') --- and the salient identifying features of a
reference should be included: title, year, volume, number, pages,
article DOI, etc.

The bibliography is included in your source document with these two
commands, placed just before the \verb|\end{document}|
command:
% \begin{lstlisting}
% \bibliography{bibfile}
% \end{lstlisting}
where ``\verb|bibfile|'' is the name, without the ``\verb|.bib|''
suffix, of the Bib\TeX{} file.

\begin{acknowledgments}

  The results were obtained within the state assignment of the Ministry of Education and Science of Russia, the project ``Methods and technologies of a cloud-based service-oriented digital platform for collecting, storing and processing large volumes of multi-format interdisciplinary data and knowledge based on the use of artificial intelligence, a model-driven approach and machine learning'', No.~FWEW-2021-0005 (State registration No.~121030500071-2).

  The results obtained with the use of the network infrastructure of Telecommunication center of collective use ``Integrated information-computational network of Irkutsk scientific-educational complex'' (\url{http://net.icc.ru}).

The~R\&D on GIS-viewer of faults involved the Centre of Geodynamics and Geochronology equipment at the Institute of the Earth's Crust, Siberian Branch of the Russian Academy of Sciences (grant No.~075-15-2021-682).   This study direction was partially carried out within the basic budgetary research project ``Modern geodynamics, mechanisms of destruction of the lithosphere and hazardous geological processes in Central Asia'', No.~FWEF-2021-0009.

The~development of the~infrastructure for Mothur command transformation to Rapidminer dataflow diagram is supported by the project of Irkutsk scientific center of Siberian branch of Russian Academy of sciences, grant No 4.2.

\end{acknowledgments}
%%
%% Define the bibliography file to be used
% \bibliography{sample-ceur}

\begin{thebibliography}{99}

\bibitem{lod} C.~Bizer, T.~Heath, T.~Berners-Lee, Linked Data -- The Story So Far. Int. J. Semantic Web Inf. Syst., 5 (2009), pp.~1--22. \doi{10.4018/jswis.2009081901}

\bibitem{lunina} O.~V.~Lunina.  The digital map of the Pliocene–Quaternary crustal faults in the Southern East Siberia and the adjacent Northern Mongolia. Geodynamics \& Tectonophysics.  V.~7(3) (2016). pp.~407-434. (in Russian) \doi{10.5800/GT-2016-7-3-0215}

\bibitem{hogan} A.~Hogan, E.~Blomqvist, M.~Cochez, C.~D’Amato \emph{et al}. Knowledge Graphs, 2020. \url{https://arxiv.org/abs/2003.02320v5}

\bibitem{afs} A.~A.~Gladkov, O.~V.~Lunina. Cartographic service ``Activetectonics''. \url{http://activetectonics.ru/} (access date: 20-Sep-2021)

\bibitem{foss} M.~Leidig, R.~Teeuw. Free software: A review, in the context of disaster management. International Journal of Applied Earth Observation and Geoinformation, 42 (2015), pp.~49-56. \doi{10.1016/j.jag.2015.05.012}.

\bibitem{lgd} C.~Stadler, J.~Lehmann, K.~Höffner, S.~Auer. LinkedGeoData: A core for a web of spatial open data. Semantic Web 3 (2012) 333–354. \doi{10.3233/SW-2011-0052}

\bibitem{geolink} M.~Cheatham, A.~Krisnadhi, R.~Amini, P.~Hitzler, \emph{et al}. The GeoLink knowledge graph, Big Earth Data, 2:2 (2018), pp.~131-143. \doi{10.1080/20964471.2018.1469291}

\bibitem{abid} T.~Abid, H.~Zarzour. Integrating linked open data in geographical information system. Procs. of. International Conference on Information Technology for Organization Development. Oct 19-20, 2014, University of Tebessa, Tebessa, Algeria (2014).

\bibitem{iwaniak1} A.~Iwaniak, I.~Kaczmarek, M.~Strzelecki, J.~Lukowicz, P.~Jankowski. Enriching and improving the quality of linked data with GIS. Open Geosciences, Vol.~8, 1~(2016) pp.~323-336. \doi{10.1515/geo-2016-0020}

\bibitem{iwaniak17} A.~Iwaniak, M.~Leszczuk, M.~Strzelecki, F.~Harvey, I.~Kaczmarek. A Novel Approach for Publishing Linked Open Geodata from National Registries with the Use of Semantically Annotated Context Dependent Web Pages. International Journal of Geo-Information. 6, 252 (2017). \doi{10.3390/ijgi6080252}

\bibitem{zont19} E.~Cherkashin, A.~Shigarov, V.~Paramonov. Representation of MDA transformation with logical objects. International Multi-Conference on Engineering, Computer and Information Sciences (SIBIRCON), Novosibirsk, Russia. (2019) 0913--0918 \doi{10.1109/SIBIRCON48586.2019.8958008}

\bibitem{authoring} E.~Cherkashin, A.~Shigarov, V.~Paramonov, A.~Mikhailov, Digital archives supporting document content inference, Procs. of 42-nd International Convention on Information and Communication Technology Electronics and Microelectronics (MIPRO), May, 20–24, 2019. pp. 1037-1042. \doi{10.23919/MIPRO.2019.8757196}

\bibitem{gisviewer} E.~Cherkashin, O.~Lunina, L.~Demyanov, A.~Tsygankov. Web-GIS viewer for active faults data represented as a knowledge graph. Proceedings for 4th Scientific-practical Workshop Information Technologies: Algorithms, Models, Systems. Irkutsk, Russia, September 14, 2021. CEUR-WS.org, online \url{http://ceur-ws.org/Vol-2984/paper8.pdf}

\end{thebibliography}
\section{Online Resources}

The sources for the viewer are being developed at Github, URL:~\url{https://github.com/De17eon/GRL}.

\end{document}

%%
%% End of file


%%% Local Variables:
%%% mode: latex
%%% TeX-master: "paper-aicts-cherkashin.tex"
%%% End:
